\addcontentsline{toc}{subsection}{La astronomía como ciencia}
\subsection*{La astronomía como ciencia}

En la revisión literaria podemos encontrar las siguientes definiciones de astronomía:

La astronomía es la ciencia natural del universo, en su
concepto más general. La astronomía
se dedica a estudiar las posiciones, distancias, movimientos, estructura y evolución de los astros y para ello se basa casi exclusivamente en la información
contenida en la radiación electromagnética o de partículas que alcanza al observador \cite{conceptos2009}


Es una disciplina que nos abre los ojos, da contexto a nuestro lugar en el universo y puede remodelar la idea de cómo vemos el mundo \cite{iau2025}.

En base a las definiciones presentadas, se puede afirmar que la astronomía se trata de una ciencia y disciplina fundamental
para comprender la estructura y evolución del universo, permitiendo grandes avances en conocimiento del universo.

La exploración de dicha ciencia, ha permitido comprender la formación y evolución de los cuerpos rocosos que están en nuestro sistema solar, así como el impacto de eventos astronómicos en la geología terrestre.

Dentro de ese campo, la luna es uno de los cuerpos más estudiados debido a su cercanía con la Tierra, además su influencia en nuestro planeta
como en aspectos físicos como formación de mareas hasta aspectos biológicos han sido tema de numerosos estudios.

\addcontentsline{toc}{subsection}{El fenómeno de las fases lunares}
\subsection*{El fenómeno de las fases lunares}


Estas fases están determinadas por las posiciones relativas cambiantes de la Luna, la Tierra y el Sol, que determinan qué parte de la superficie de la Luna está iluminada desde la perspectiva de la Tierra \cite{gullari2025}

En la tabla 4.1 se encuentran se aprecia detalladamente propiedades de este fenómeno astronómico:

\begin{table}[h]
    \centering
    \renewcommand{\arraystretch}{1.2} % Ajusta el espaciado entre filas
    \begin{tabular}{|l|c|p{7cm}|}
        \hline
        \textbf{Fase} & \textbf{\% Iluminación} & \textbf{Características Observables} \\
        \hline
        Luna Nueva & 0\% & Invisible desde la Tierra. \\
        \hline
        Luna Creciente & 1-49\% & Franja de luz en el lado derecho. \\
        \hline
        Cuarto Creciente & 50\% & La mitad derecha es visible. \\
        \hline
        Luna Gibosa Creciente & 51-99\% & Brillo creciente, casi llena. \\
        \hline
        Luna Llena & 100\% & Totalmente iluminada. \\
        \hline
        Luna Gibosa Menguante & 51-99\% & Brillo decreciente, luz en el lado izquierdo. \\
        \hline
        Cuarto Menguante & 50\% & La mitad izquierda es visible. \\
        \hline
        Luna Menguante & 1-49\% & Media luna antes de la luna nueva. \\
        \hline
    \end{tabular}
    \caption{Fases Lunares y Características Resumidas}
    \label{tabla:fases_lunares}
\end{table}

Para que se complete un ciclo completo de fases, la luna tarda aproximadamente 29,53 días. Esto ha sido fundamental en la creación de calendarios y ha influido en diversas culturas para determinar
festividades y actividades agrícolas.

\addcontentsline{toc}{subsection}{Evolución científica del estudio de la Luna}
\subsection*{Evolución científica del estudio de la Luna}

En el siglo XX, los estudios de la luna alcanzó su punto más exponencial con la misión de Apollo 11 en 1969, cuando
los astronautas Neil Armstrong y Buzz Aldrin se convirtieron en los primeros seres humanos en pisar la Luna.

Recientemente, misiones como  Lunar Reconnaissance Orbiter (LRO) y Chandrayaan-2 \cite{Singh2022} han proporcionado datos para
estudiar presencia de hielo en los polos lunares lo que permite explorar gracias a avances de la tecnología y simulaciones, las
características internas del satélite.

Se concluye en esta sección del área de investigación que la luna ha sido objeto de estudio y fascinación a lo largo de la historia. Sus fases han sido fundamentales para la creación de calendarios, mientras que su exploración ha impulsado el desarrollo tecnológico y científico. Comprender su fenómeno astronómico no solo amplía el conocimiento del sistema solar, sino que también sienta las bases para futuras misiones espaciales que podrían utilizar la Luna como punto de partida para la exploración interplanetaria.


\addcontentsline{toc}{subsection}{Importancia de la Simulación en el Estudio de Fenómenos Astronómicos}
\subsection*{Importancia de la Simulación en el Estudio de Fenómenos Astronómicos}

\addcontentsline{toc}{subsubsection}{Ámbito científico}
\subsubsection*{Ámbito científico}


En este ámbito, el grupo de investigación GPLA se desarrolló una serie de códigos escritos en
Python para la simulación de ondas electrostáticas uni- y bidimensionales \cite{ondas}. Esto se hizo para simular condiciones del plasma donde
no solo existan haces o haz y una especie de electrones en el universo.

Además, la importancia de la simulación proporciona la capacidad de modelary predecir el comportamiento de sistemas planetarios\cite{perryman}.

La idoneidad de Python como lenguaje de aplicaciones ha sido posible gracias a los grandes
progresos logrados en la última década en la mejora de las herramientas que permiten utilizarlo
de manera efectiva para manipular datos astronómicos \cite{green}.

En base a la literatura, se puede evidenciar que hay una gran importancia de realizar simulaciones en el estudio de fenómenos astronómicos
gracias a software vérsatil o incluso con la integración de lenguajes de programación como lo fue Python para la manipulación de grandes volumenes
de datos y su posterior simulación. La evolución de los lenguajes de programación ayudan a los científicos a analizar, estudiar, y representar datos astronómicos 
para modelar comportamientos astronómicos y obtener conclusiones que permitan desarrollar teorías científicas.



\addcontentsline{toc}{subsubsection}{Ámbito educativo}
\subsubsection*{Ámbito educativo}

El simulador Stellarium fue utilizado en cada sesión de 60 minutos por la educadora en la primera y tercera unidad. Se eligió esta herramienta ya que muestra el cielo nocturno y su evolución a lo largo del tiempo. En la primera unidad, esta herramienta permitió dar a conocer los niños y niñas las trayectorias que siguen los planetas en el cielo y los meses en los que pueden ser vistos. También señalar los movimientos de la luna y sus fases \cite{PerezLisboa2020}.

Los autores del trabajo previamente descrito, afirman que Las TIC, son un medio didáctico recreativo donde el profesor y el estudiante no solo interactúan con objetos inanimados, si no que se les puede generar movimiento, lo que es muy divertido para aprender y enseñar de una manera didáctica.

El uso de la Realidad Aumentada (RA) como una herramienta innovadora en la enseñanza de la astronomía. Desarrollado por el Instituto Nacional de Astrofísica de Italia (INAF), este proyecto implementa aplicaciones de RA para enriquecer la experiencia educativa, permitiendo a los estudiantes interactuar con modelos tridimensionales y obtener información adicional sobre diversos fenómenos astronómicos. Estas aplicaciones han sido distribuidas en escuelas y al público general a través de EduINAF, la revista en línea dedicada a la educación y divulgación científica \cite{nasa}. 
La integración de RA en el aprendizaje ha demostrado aumentar la motivación y comprensión de los estudiantes, ofreciendo oportunidades educativas únicas y fomentando una experiencia de aprendizaje más inmersiva y atractiva.

La revisión literaria nos indica que la integración de software de simulación en las escuelas, demuestra que se logra enseñar temas de astronomía de manera inmersa y a la vez fascinante lo cual hace
que el aprendizaje mediante simulación sea una opción recurrida a explicar fenómenos astronómicos con mayor claridad y didáctica.
