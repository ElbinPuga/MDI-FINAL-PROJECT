\addcontentsline{toc}{subsection}{Lenguaje de programación}
\subsection*{Lenguaje de programación}
Para el desarrollo de la propuesta del prototipo, se implemetará el lenguaje de programación Python debido 
a su ámbito en área científica y su integración en la Web.

\begin{figure}[H]
    \includegraphics[scale = 0.80]{Imagenes/python-logo.png}
    \centering
    \caption{Logo de Python}{Fuente: Adaptado de Internet}
\end{figure}

\begin{itemize}
    \item \textbf{Facilidad de uso:} Python es un lenguaje de alto nivel con una sintaxis clara y legible, lo que facilita el desarrollo y mantenimiento del código.
    \item \textbf{Disponibilidad de librerías:} Posee una amplia gama de librerías especializadas en cálculo astronómico, visualización y desarrollo web.
    \item \textbf{Comunidad de soporte:} Python cuenta con una de las comunidades de desarrolladores más grandes, lo que garantiza una amplia documentación y soporte técnico.
\end{itemize}

\addcontentsline{toc}{subsection}{Librerías y Frameworks}
\subsection*{Librerías y Frameworks}

El desarrollo del prototipo utilizará diversas \textbf{librerías y frameworks} que contribuyen a la precisión y eficacia de la simulación:

\addcontentsline{toc}{subsection}{Cálculo y Simulación Astronómica}
\subsection*{Cálculo y Simulación Astronómica}
\begin{itemize}
    \item \textbf{Skyfield:} Permite calcular posiciones astronómicas con alta precisión, incluyendo las fases de la Luna basadas en efemérides oficiales de la NASA.
    \item \textbf{NumPy:} Facilita operaciones numéricas eficientes para manejar datos astronómicos.
\end{itemize}

\addcontentsline{toc}{subsection}{Visualización de Datos}
\subsection*{Visualización de Datos}
\begin{itemize}
    \item \textbf{Matplotlib:} Utilizado para generar gráficos que representen la evolución de las fases de la Luna de manera precisa y clara.
    \item \textbf{Plotly:} Posible alternativa para gráficos interactivos en la interfaz web.
\end{itemize}

\addcontentsline{toc}{subsection}{Desarrollo Web e Interfaz Gráfica}
\subsection*{Desarrollo Web e Interfaz Gráfica}
\begin{itemize}
    \item \textbf{Flask:} Framework ligero en Python para manejar la lógica del servidor y comunicación con la interfaz de usuario.
    \item \textbf{HTML, CSS y JavaScript:} Tecnologías fundamentales para la creación de la interfaz web interactiva.
\end{itemize}

\addcontentsline{toc}{subsection}{Generación de Reportes}
\subsection*{Generación de Reportes}
\begin{itemize}
    \item \textbf{ReportLab:} Permite la generación de archivos PDF para que los usuarios descarguen información detallada sobre las fases de la Luna.
\end{itemize}

\addcontentsline{toc}{subsection}{Justificación del Uso de Estas Herramientas}
\subsection*{Justificación del Uso de Estas Herramientas}
\begin{itemize}
    \item \textbf{Skyfield} asegura precisión astronómica al basarse en datos de efemérides.
    \item \textbf{Matplotlib y Plotly} permiten representar gráficamente los cambios en las fases lunares de manera comprensible.
    \item \textbf{Flask + HTML/CSS/JavaScript} garantizan una interfaz intuitiva y accesible desde cualquier navegador.
    \item \textbf{por definir***} proporciona material educativo en PDF, facilitando la revisión posterior por parte de los estudiantes.
\end{itemize}

\addcontentsline{toc}{subsection}{Plataformas y Entornos de Desarrollo}
\subsection*{Plataformas y Entornos de Desarrollo}

Para el desarrollo del prototipo, se utilizarán las siguientes herramientas:

\begin{itemize}
    \item \textbf{Entorno de Desarrollo Integrado (IDE):} Se usa \textbf{VS Code} o \textbf{PyCharm}, ambos con soporte para Python y depuración eficiente.
    \item \textbf{Control de Versiones:} Git y GitHub serán empleados para gestionar cambios en el código y colaboración entre desarrolladores.
    \item \textbf{Entorno Virtual:} Se utilizará \texttt{venv} o \texttt{conda} para administrar dependencias y evitar conflictos entre librerías.
\end{itemize}

\addcontentsline{toc}{subsection}{Beneficios del Uso de Software de Código Abierto}
\subsection*{Beneficios del Uso de Software de Código Abierto}
\begin{itemize}
    \item Facilita la replicación y ampliación del proyecto por otros investigadores o educadores.
    \item Reduce costos al evitar el uso de licencias propietarias.
    \item Permite contribuciones de la comunidad para mejorar la simulación.
\end{itemize}

