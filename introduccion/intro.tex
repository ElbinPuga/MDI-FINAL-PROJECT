El campo de la astronomía es una ciencia que ha fascinado a la humanidad desde tiempos inmemoriales,
siendo la Luna uno de los objetos más estudiados y admirados. Las fases de la Luna, en particular,
no solo han sido motivo de observación y mitología, sino que también representan un fenómeno complejo
cuyo entendimiento resulta fundamental para diversas aplicaciones científicas y educativas.

A pesar de la relevancia de este fenómeno, la enseñanza tradicional de la astronomía en muchos contextos escolares
adolece de métodos interactivos y actualizados que faciliten la comprensión de conceptos tan abstractos como la mecánica celeste. 
Esta situación ha generado una brecha entre el conocimiento teórico y la práctica, afectando la calidad del aprendizaje en el área.

Ante esta problemática, la presente investigación propone el desarrollo de un prototipo de simulación computacional interactivo orientado 
a representar las fases de la Luna, con el fin de mejorar la comprensión del fenómeno y promover una educación más dinámica y participativa.
