\newpage
\addcontentsline{toc}{section}{Formulario de verificación de entrega de documentos}
% Titulo
{\bfseries \centering
\Universidad \\
\Facultad \\
VICEDECANATO ACADÉMICO \\
FORMULARIO DE VERIFICACIÓN DE ENTREGA DE DOCUMENTOS}

\vfill
{
\renewcommand{\arraystretch}{1.7}
% Tabla de formulario
\newcolumntype{b}{>{\hsize=.95\hsize}X} % ~95% ancho pagina
\newcolumntype{s}{>{\hsize=.05\hsize}X} % ~5% ancho pagina
\begin{tabularx}{\textwidth}{sb} % |s|b|
  % \hline
  (✔) & \begin{enumerate} \item[1.] Página de Presentación usando el formato respectivo. Nombre de la Universidad, Nombre de la Facultad, Título del Trabajo, Tipo de Trabajo, Nombre del Profesor Asesor, Integrante(s), Título a Optar, Año Elaboración. \end{enumerate} \\
  % \hline
  (✔) & \begin{enumerate} \item[2.] Formulario de Registro Oficial del tema del trabajo de graduación firmado por el (los) estudiante(s) y el profesor Asesor propuesto. \end{enumerate} \\
  % \hline
  (✔) & \begin{enumerate} \item[3.] Introducción. Incluye descripción de la situación actual, propuesto y las mejoras que el proyecto persigue. Debe incluir definición y alcance del tema, metodología y técnica de investigación a utilizar. \end{enumerate} \\
  % \hline
  (✔) & \begin{enumerate} \item[4.] Índice del Anteproyecto \end{enumerate} \\
  % \hline
  (✔) & \begin{enumerate} \item[5.] Objetivos (General y Específicos) \end{enumerate} \\
  % \hline
  (✔) & \begin{enumerate} \item[6.] Plan de Contenido propuesto para el desarrollo del proyecto. \end{enumerate} \\
  % \hline
  (✔) & \begin{enumerate} \item[7.] Bibliografía. Generalmente el mínimo son 10 referencias bibliográficas. Lo importante es que sea actualizado de acuerdo con el tema (por lo menos los últimos 5 años) \end{enumerate} \\
  % \hline
  (✔) & \begin{enumerate} \item[8.] Cronograma de actividades. Presentado en diagrama de Gantt, preferiblemente en términos de meses o semanas. Los meses o semanas no deben ser etiquetados cronológicamente. \end{enumerate} \\
  % \hline
  (✔) & \begin{enumerate} \item[9.] Herramientas de Software y Hardware que utilizar. \end{enumerate} \\
  % \hline
  ( ) & \begin{enumerate} \item[10.] Créditos Académicos Oficiales y Formulario de Verificación de Requisitos para Trabajo de Graduación (Secretaría Académica) \end{enumerate} \\
  % \hline
  ( ) & \begin{enumerate} \item[11.] Fotocopia de la constancia de matrícula del trabajo de graduación matriculado y el pago del seguro estudiantil por cada estudiante. La Matrícula debe cubrir el semestre o año académico en que se desarrolla el proyecto. \end{enumerate} \\
  % \hline
  ( ) & \begin{enumerate} \item[12.] Información del Programa de Práctica Profesional (cuando sea el caso). \end{enumerate} \\
  % \hline
\end{tabularx}
}

\vfill
REVISADO POR:\underline{\asesor} \hspace*{1em}FECHA:\underline{\dd/\mm/\aaaa}
