\addcontentsline{toc}{subsection}{Transición a la era digital}
\subsection*{Transición a la era digital}

Las simulaciones numéricas han permitido a los astrónomos estudiar procesos complejos que son inaccesibles mediante observación directa. 
Por ejemplo, con la llegada de telescopios de nueva generación como el Large Synoptic Survey Telescope (LSST) la astronomía ha entrado en la era del "Big Data". El LSST, por ejemplo, generará aproximadamente 15 terabytes de datos cada noche, lo que requiere el desarrollo de nuevas técnicas de análisis y almacenamiento de datos. Esto ha llevado al surgimiento de la astroinformática, una disciplina que combina la astronomía con la informática para manejar y analizar estos vastos conjuntos de datos\cite{huijse2014}.

También se puede encontrar que la visualización de datos astronómicos ha avanzado significativamente con el uso de tecnologías de realidad virtual (VR). Herramientas
como iDaVIE\cite{jarrett2021} permiten a los investigadores explorar datos astrofísicos en entornos inmersivos, facilitando una comprensión más intuitiva de estructuras complejas y relaciones espaciales en el cosmos. Esta metodología mejora la interpretación de datos y apoya el descubrimiento científico

Con estos antecedentes de la era digital con la integración de la computación en la astronomía ha permitido avances significativos en la simulación, análisis y visualización de fenómenos cósmicos, ampliando nuestra comprensión del universo y abriendo nuevas vías para la investigación científica.


