Se plantea una metodología investigación-acción para el desarrollo del prototipo, que combina la acción y la reflexión en un proceso cíclico e iterativo. En la fase inicial se realizará una exhaustiva investigación y análisis de los fundamentos astronómicos relacionados con las fases de la Luna, consultando fuentes académicas, libros especializados y bases de datos científicas. Durante esta etapa se identificarán las variables clave y se definirán los parámetros que deben ser simulados, permitiendo comprender a fondo tanto el fenómeno como las necesidades educativas asociadas.

Con la información recopilada, se procederá a la planificación y diseño del prototipo. Aquí se definirá la arquitectura del software y la interfaz de usuario, apoyándose en diagramas y herramientas como UML para plasmar la estructura del sistema. Se seleccionarán tecnologías y lenguajes de programación adecuados—por ejemplo, Python y JavaScript—que permitan integrar gráficos interactivos y simulaciones en tiempo real, facilitando la interacción del usuario mediante el ajuste de parámetros como el ángulo de visión o la posición relativa de la Tierra y la Luna.

La fase de acción consistirá en la implementación del prototipo, donde se integrarán algoritmos matemáticos con elementos gráficos mediante un proceso de codificación modular. Se aplicarán pruebas unitarias e integradas para garantizar la precisión y estabilidad del sistema, y se documentará de forma continua el desarrollo, estableciendo mecanismos de control de versiones que permitan incorporar mejoras de manera iterativa.

Finalmente, se llevará a cabo una fase de evaluación y retroalimentación, característica de la investigación-acción. En esta etapa, el prototipo se pondrá a disposición de expertos en astronomía y educación, tanto a nivel nacional como internacional, para recibir observaciones críticas y sugerencias. Los resultados obtenidos serán objeto de reflexión y análisis, lo que posibilitará ajustar y refinar el prototipo en nuevos ciclos de acción. Este enfoque cíclico no solo asegura la solución de los problemas identificados, sino que también fomenta la generación de nuevo conocimiento y la continua optimización del recurso didáctico.
